\documentclass[12pt]{report}
\usepackage{amsmath}
\usepackage{fullpage}
%\usepackage{indentfirst}
%\renewcommand{\baselinestretch}{1.25}

%\usepackage[pdftex]{graphicx}
\newcommand{\CC}{C\nolinebreak\hspace{-.05em}\raisebox{.4ex}{\tiny\bf +}\nolinebreak\hspace{-.10em}\raisebox{.4ex}{\tiny\bf +}}
\def\CC{{C\nolinebreak[4]\hspace{-.05em}\raisebox{.4ex}{\tiny\bf ++}}}

\usepackage{color}
\newcommand{\new}[1]{\textcolor{red}{#1}}

\usepackage[export]{adjustbox}

\textheight9in

\begin{document}  
\begin{center}
\Large{Anisotropy Interaction}
\end{center}

Consider the single ion anisotropy term in the spin Hamiltonian
\begin{equation}
H = \sum_i D_i \,(\boldsymbol{S}_i \cdot \hat{n})^2
\end{equation}

where $\hat{n} = \left(u_x,u_y,u_z\right)$ is a unit vector and the subscript
$i$ indicates an unique sublattice. $D_i$ is negative for easy-axis anisotropy
and positive for easy-plane anisotropy. Solving this expression, we consider
an alternative and more general expression for the single ion anisotropy.
 
\begin{equation}
\label{anisotropy}
H = \sum_i \boldsymbol{S}_i \cdot \overline{\overline{A}}_i \cdot \boldsymbol{S}_i
\end{equation}

Comparing these two equations we determined these two expressions are
equal when 

\begin{equation}
\overline{\overline{A}}_i = D_i \left( 
\begin{matrix}
u_x^2 & u_x u_y & u_x u_z \\
u_x u_y & u_y^2 & u_y u_z \\
u_x u_z & u_y u_z & u_z^2
\end{matrix} 
\right)
\end{equation}

which is clearly a symmetric matrix. The zeroth order energy can be calculated from eqn. 
\ref{anisotropy} by treating the spins classically. To calculate the first and second order terms,
in the Holstein-Primakoff expansion we rotate into the reference frame of each moment, 
$\overline{\boldsymbol{S}}_i = \overline{\overline{U}}_i \cdot \boldsymbol{S}_i $ and calculate 
terms in the form of a constant times 

\begin{equation}
\left( \overline{\overline{U}}^{-1}_i \boldsymbol{S}_i \right)_{\alpha}
\left( \overline{\overline{U}}^{-1}_i \boldsymbol{S}_i \right)_{\beta}
\end{equation}

To simplify the notation, from now on I will drop the subscript $i$ and define $\overline{\overline{V}} = \overline{\overline{U}}^{-1}$. 


\begin{align*}
 (\overline{\overline{V}} \boldsymbol{S})_{\alpha} (\overline{\overline{V}} \boldsymbol{S})_{\beta} = \,\,\,
 &V_{\alpha,0} V_{\beta,0} S_x S_x + V_{\alpha,0} V_{\beta,1} S_x S_y +V_{\alpha,0} V_{\beta,2} S_x S_z \\
 &V_{\alpha,1} V_{\beta,0} S_y S_x + V_{\alpha,1} V_{\beta,1} S_y S_y +V_{\alpha,1} V_{\beta,2} S_y S_z \\
  &V_{\alpha,2} V_{\beta,0} S_z S_x + V_{\alpha,2} V_{\beta,1} S_z S_y +V_{\alpha,2} V_{\beta,2} S_z S_z 
 \end{align*} 

The first order terms involve $S_x S_z, S_y S_z, S_z S_x,$ or $S_z S_y $ They are 

\begin{equation}
V_{\alpha,0} V_{\beta,2} S_x S_z + V_{\alpha,1} V_{\beta,2} S_y S_z +  V_{\alpha,2} V_{\beta,0} S_z S_x + V_{\alpha,2} V_{\beta,1} S_z S_y
\end{equation}

applying the Holstein-Primakoff formalism, this becomes

\begin{equation}
 (\overline{\overline{V}} \boldsymbol{S})_{\alpha} (\overline{\overline{V}} \boldsymbol{S})_{\beta} = \sqrt{\frac{S^3}{2}} \left( \nu^* a_i + \nu a_i^{\dagger} \right)
\end{equation}
where
\begin{equation}
\nu\left(\alpha,\beta\right) = V_{\alpha,2} V_{\beta,0} + V_{\alpha,0} V_{\beta,2} +i \left (  V_{\alpha,2} V_{\beta,1} + V_{\alpha,1} V_{\beta,2} \right)
\end{equation}

The second order terms include all that remain. They are 
\begin{equation}
 V_{\alpha,0} V_{\beta,0} S_x S_x + V_{\alpha,0} V_{\beta,1} S_x S_y
 + V_{\alpha,1} V_{\beta,0} S_y S_x + V_{\alpha,1} V_{\beta,1} S_y S_y 
  +V_{\alpha,2} V_{\beta,2} S_z S_z 
\end{equation}

Written in terms of spin operators $S_+,S_-,S_z$, this expression becomes

\begin{align}
 (\overline{\overline{V}} \boldsymbol{S})_{\alpha} (\overline{\overline{V}} \boldsymbol{S})_{\beta}&=\frac{1}{2} \left( \eta^* S_+ S_+ +  \eta S_- S_- + \zeta^* S_+ S_- + \zeta S_-S_+ \right) + V_{\alpha,2} V_{\beta,2} S_z S_z \\
 \zeta\left(\alpha,\beta\right)  &= V_{\alpha,0} V_{\beta,0} + V_{\alpha,1} V_{\beta,1} +i \left (  V_{\alpha,1} V_{\beta,0} - V_{\alpha,0} V_{\beta,1} \right)\\
 \eta\left(\alpha,\beta\right) &= V_{\alpha,0} V_{\beta,0} - V_{\alpha,0} V_{\beta,1} +i \left (  V_{\alpha,1} V_{\beta,0} + V_{\alpha,1} V_{\beta,1} \right)
\end{align} 

Applying the Holstein-Primakoff formalism and taking the Fourier transform, this becomes

\begin{equation}
S \left( \eta^* a_{-k} a_k +  \eta a_{k}^{\dagger} a_{-k}^{\dagger}  + \left(\zeta^* - 2\,V_{\alpha,2} V_{\beta,2}\right) a_k^{\dagger} a_k + \zeta a_k a_k^{\dagger} \right) \\
\end{equation}

The L Matrix (eqn. A4 in Phys. Rev. B  87 134416 (2013)), is divided into four quadrants with 
terms $a_{-k} a_k $ (lower left), $a_k^{\dagger} a_{-k}^{\dagger}$ (upper right), $a_k a_k^{\dagger}$ (lower right), and $a_k^{\dagger} a_k$ (upper left).
Including the single ion anisotropy modifies the L matrix as follows:

(lower left)
\begin{equation}
L(r+M,r) = D_r S_r  \sum_{\alpha,\beta} A_r(\alpha,\beta) \eta_r(\alpha,\beta)^* 
\end{equation}

(upper right)

\begin{equation}
L(r,r+M) = D_r S_r  \sum_{\alpha,\beta} A_r(\alpha,\beta) \eta_r(\alpha,\beta) 
\end{equation}
 
 (upper left) = (lower right)
 
 \begin{equation}
L(r,r)  = L(r+M,r+M) =  D_r S_r  \sum_{\alpha,\beta} A_r(\alpha,\beta)\left( RE(\zeta_r(\alpha,\beta)) - \left(V_r\right)_{\alpha,2} \left(V\right)_{\beta,2} \right)
 \end{equation}
 
 Note:
 
 1) None of these terms explicitly depend on $k$ and therefore don't need to be recalculated at each $k$-point.
 
 2) One could rewrite these expressions to use only the upper or lower diagonal of A$_r$, reducing the number of terms in the sum from 9 to 6. However, these 
 terms are only calculated once and stored for the remainder or the calculation and any performance benefit too small to measure.
 





\end{document}
